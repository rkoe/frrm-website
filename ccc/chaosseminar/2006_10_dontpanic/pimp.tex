%=========================================================================
% DocumentClass-Varianten: Standard-A4, Standard-A5, Verkleinern auf A5:
%-------------------------------------------------------------------------
% \documentclass[11pt,a4paper,twoside]{article}
\documentclass{beamer}
\usepackage{listings}
\lstset{language=bash}
\mode<presentation>
{
% \setbeamertemplate{background canvas}[vertical shading][bottom=red!10,top=blue!10]

\usetheme{Madrid}
% \usetheme{Boadilla}
\usefonttheme[onlysmall]{structurebold}
\useoutertheme[footline=authortitle]{miniframes}
\useinnertheme[shadow]{rounded}
}

\usepackage[T1]{fontenc}
% \setbeamercovered{transparent}

% \pgfdeclareimage[width=1cm]{debian-logo}{openlogo}
% \logo{\pgfuseimage{debian-logo}}
\pgfdeclareimage{scsi-disk-old}{scsi-old}
\pgfdeclareimage{ide-disk-old}{ide-old}
\pgfdeclareimage{ide-gross}{ide-new-gross}
\pgfdeclareimage{ide}{ide-new}
\pgfdeclareimage{panic}{dont}

\title{Pimp my dontpanic}
\author{Ulrich Dangel Alexander Bernauer Tobias Rehfeld}

\begin{document}

\begin{frame}
    \titlepage
    \begin{center}
    \pgfuseimage{panic}
    \end{center}
\end{frame}

\begin{frame}
    \frametitle{�bersicht}
    \tableofcontents
\end{frame}

\section{Begriffskl�rung}

\begin{frame}
    \frametitle{Software Raid}
    \begin{itemize}
        \item Redudant Array of Inexpensive Disks
        \item Verschiedene Betriebsmodi
            \begin{itemize}
                \item 0 - Stripping
                \item 1 - Mirroring
                \item 5 - Performance + Parit�t
            \end{itemize}
        \item Wird vom Kernel automatisch erkannt
    \end{itemize}
\end{frame}
\begin{frame}
    \frametitle{lvm - Logical Volume Manager}
    \begin{itemize}
        \item Abstrahierung von Block Devices
        \item Block Device -> Physical Volume
        \item Volume Gruppen, bestehen min. 1 Physical Volume
        \item Logical Volume in Volume Gruppen
        \item Keine automatische Erkennung im Kern, initrd
    \end{itemize}
\end{frame}

\begin{frame}
    \frametitle{initrd - Initial Ramdisk}
    \begin{itemize}
        \item Bootloader l�dt den Kernel und die initrd
        \item Ramdisk wird als / rw gemountet
        \item /linuxrc wird ausgef�hrt
        \item Beim beenden wird /sbin/init gestartet
    \end{itemize}
\end{frame}

\begin{frame}
    \frametitle{pxe - Preboot Execution Environment}
    \begin{itemize}
        \item dhcp Request
        \item Dateiname + Server stehen in der Antwort
        \item Dateiname wird per tftp geladen
        \item L�dt weitere Teile nach
    \end{itemize}
\end{frame}


\section{Orginal Setup}
\subsection{Hardware}
\begin{frame}
    \frametitle{Eingebaute Hardware}
    \begin{itemize}
        \item 4x 9GB SCSI Platten
        \item 1x 160 GB IDE Platte
    \end{itemize}
\end{frame}
\subsection{Partitionen}
\begin{frame}
    \frametitle{Partionen - SCSI}
    \begin{itemize}
        \item Die SCSI Platten sind alle gleich formatiert
        \pgfuseimage{scsi-disk-old}
                \item 1. Partion ca.  15 MB - Raid 1 - /boot
                \item 2. Partition ca. 2 GB - Raid 1 - /rescue
                \item 3. Partition ca. 7 GB - Raid 5 - /
    \end{itemize}
\end{frame}

\begin{frame}
    \frametitle{Partitionen - IDE}
    \begin{itemize}
        \item Eine grosse Partition 
        \pgfuseimage{ide-disk-old}
        \item 1. Partion ca. 160 GB /data
    \end{itemize}
\end{frame}

\section{Geplantes Setup}
\subsection{Hardware}
\begin{frame}
    \frametitle{Neue Hardware}
    \begin{itemize}
        \item 2x320 GB IDE Platten
        \item 1x160 GB IDE Platten
    \end{itemize}
\end{frame}
\subsection{Partitionen}
\begin{frame}
    \frametitle{Partitionen - �berblick}
            \pgfuseimage{ide}
            \pgfuseimage{ide-gross}
            \pgfuseimage{ide-gross}
\end{frame}
\begin{frame}
    \frametitle{Partitionen - Kleine Platte}
    \pgfuseimage{ide}
    \begin{itemize}
        \item 1 ca.  50 MB - Raid 1 - /boot
        \item 2 ca.   5 GB - Raid 1 - /rescue
        \item 3 ca.  30 GB - Raid 5 - lvm - system 
        \item 4 ca. 125 GB - Raid 5 - lvm - archiv
    \end{itemize}
\end{frame}

\begin{frame}
    \frametitle{Partitionen - Grosse Platte}
    \pgfuseimage{ide-gross}
    \begin{itemize}
        \item 1 ca.  50 MB - Raid 1 - /boot
        \item 2 ca.   5 GB - Raid 1 - /rescue
        \item 3 ca.  30 GB - Raid 5 - lvm - system 
        \item 4 ca. 125 GB - Raid 5 - lvm - archiv
        \item 5 ca. 160 GB - Raid 1 - lvm - ?
    \end{itemize}
\end{frame}

\section{Endg�ltiges Setup}
\begin{frame}
    \frametitle{Die Unendliche Geschichte - 1}
               Downtime: 0h\\
               ETA: 6h, inkl. Probleme

               \begin{itemize}
                   \item Rechner ausgeschaltet
                       \pause
                   \item Alles ausgebaut, neuer Controller eingebaut, Platten angeschlossen
                       \pause
                   \item Bildschirm schwarz
               \end{itemize}
\end{frame}

\begin{frame}
    \frametitle{Die Unendliche Geschichte - 2}
               Downtime: 1h\\
               ETA: 6h, inkl. kleinerer Probleme

               \begin{itemize}
                   \item Alte Konfiguration wieder hergestellt 
                       \pause
                   \item Alles ausgebaut, neuer Controller ausgebaut, Platten ausgebaut 
                       \pause
                   \item Bildschirm schwarz
               \end{itemize}
\end{frame}

\begin{frame}
    \frametitle{Die Unendliche Geschichte - 3}
               Downtime: 2h\\
               ETA: 6h, ohne Probleme

               \begin{itemize}
                   \item Problem mit Geld bewerfen
                       \pause
                   \item Neue CPU, neues Board, neues Ram
                       \pause
                   \item Problem gibt hoffentlich nach
               \end{itemize}
\end{frame}

\begin{frame}
    \frametitle{Die Unendliche Geschichte - 4}
               Downtime: 4h\\
               ETA: 6h, ohne Probleme

               \begin{itemize}
                   \item Board eingebaut
                       \pause
                   \item Alles angeschlossen
                       \pause
                   \item Dauernd cmos Fehler
               \end{itemize}
\end{frame}

\begin{frame}
    \frametitle{Die Unendliche Geschichte - 5}
               Downtime: 6h\\
               ETA: 8h, ohne Probleme

               \begin{itemize}
                   \item Anderes Netzteil ausprobiert
                       \pause
                   \item Ram getauscht
                       \pause
                   \item Dauernd cmos Fehler
                       \pause
                   \item Neues Board kaufen, nach 20 Uhr - warten.
               \end{itemize}
\end{frame}


\begin{frame}
    \frametitle{Die Unendliche Geschichte - 6}
               Downtime: >16h\\
               ETA: 8h, ohne Probleme

               \begin{itemize}
                   \item Neues Board mit mehr SATA Anschl�sse ;)
                       \pause
                   \item Keine Cmos Fehler mehr \alert{HEUREKA!}
                       \pause
                   \item Selbstgebauter Kernel bootet nicht, nicht debuggt.
                       \pause
                   \item Standard grml Kernel genommen
               \end{itemize}
\end{frame}

\begin{frame}[fragile]
    \frametitle{Die Unendliche Geschichte - $\infty$}
               Downtime: >17h\\
               ETA: 8h, ohne Probleme

               \begin{itemize}
                   \item Netzwerktreiber fehlt im netboot Package
                       \pause
                   \item Treiber suchen
                       \pause
                   \item Gigabit Karte in ethernet 100 \ldots
                       \pause
                   \item Bug in der initrd von Grml
                       \pause
                    \begin{lstlisting}
modprobe unionfs && UNIONFS=yes
                    \end{lstlisting}
               \end{itemize}
\end{frame}

\begin{frame}
    \frametitle{Die Unendliche Geschichte - $\infty$}
               Downtime: >24h\\
               ETA: 6h, ohne Probleme

               \begin{itemize}
                   \item Bleibt beim starten von init h�ngen
                       \pause
                   \item hm\ldots
                       \pause
                   \item Alles manuell gemacht
                       \pause
                   \item Rescue System steht
               \end{itemize}
\end{frame}

\begin{frame}
    \frametitle{Die Unendliche Geschichte - $\infty$}
               Downtime: LANG :(\\
               ETA: 2h, ohne Probleme

               \begin{itemize}
                   \item Altes System kopiert
                       \pause
                   \item Gestartet
                       \pause
                   \item Dienste wieder hergestellt
                       \pause
                   \item Tut. echo tut | mail \&\& Kino
               \end{itemize}
\end{frame}


\begin{frame}
    \frametitle{Die Unendliche Geschichte - $\infty$}
               Downtime: LANG :(\\
               ETA: 0h

               \begin{itemize}
                   \item SMS mit Inhalt: dontpanic steht.
                       \pause
                   \item Hingefahren
                       \pause
                   \item Monitor schwarz
                       \pause
                   \item gestartet, stress test
               \end{itemize}
\end{frame}

\begin{frame}
    \frametitle{Die Unendliche Geschichte - $\infty$}
               Downtime: LANG :(\\
               ETA: 0h

               \begin{itemize}
                   \item Funktioniert alles
                       \pause
                   \item Teilweise komisches Verhalten
                       \pause
                   \item Wird heiss, st�rzt aber nicht ab
                       \pause
                   \item Ok, geht: echo tut | mail \ldots
               \end{itemize}
\end{frame}

\begin{frame}
    \frametitle{Die Unendliche Geschichte - $\infty$}
               Downtime: 2 Tage mit Unterbrechung\\
               ETA: 2h

               \begin{itemize}
                   \item Dontpanic steht am n�chsten Morgen
                       \pause
                   \item Sehr komisches Verhalten
                       \pause
                   \item memtest kein Problem
                       \pause
                   \item memtest auf dem Notebook kein Problem
                       \pause
                   \item Erst mal schlafen.
               \end{itemize}
\end{frame}

\begin{frame}
    \frametitle{Die Unendliche Geschichte - $\infty$}
               Downtime: 2 Tage mit Unterbrechung\\
               ETA: 1h

               \begin{itemize}
                   \item Cmos Fehler im BIOS 
                       \pause
                   \item Sehr sehr komisches Verhalten
                       \pause
                   \item Tut gar nichts mehr
                       \pause
                   \item Bootet nicht mehr 
                       \pause
                   \item Warscheinlich CPU, aber nach 20 Uhr
               \end{itemize}
\end{frame}

\begin{frame}
    \frametitle{Die Unendliche Geschichte - $\infty$}
               Downtime: 3 Tage mit Unterbrechung\\
               ETA: xh

               \begin{itemize}
                   \item Alte Festplatten besorg
                   \item Neue CPU gekauft
                       \pause
                   \item Neue CPU eingebaut, gleicher Fehler
                       \pause
                   \item DNS Server hingestellt, Webseiten laufen wieder
               \end{itemize}
\end{frame}



\begin{frame}
    \frametitle{Die Unendliche Geschichte - $\infty$}
               Downtime: 5 Tage mit Unterbrechung\\
               ETA: 8h, ohne Probleme

               \begin{itemize}
                   \item Neues Board, CPU, Ram gekauft
                       \pause
                   \item Eingebaut
                       \pause
                   \item Problem unter Geld zusammengebrochen, funktioniert
               \end{itemize}
\end{frame}

\begin{frame}
    \frametitle{Die Unendliche Geschichte - $\infty$}
               Downtime: 5 Tage mit mehreren Unterbrechung\\
               ETA: ---

               \begin{itemize}
                   \item 1 Festplatte kaputt
                       \pause
                   \item 2. Auff�llig
                       \pause
                   \item Zu sp�t, n�chster Tag Feiertag, aber l�uft
               \end{itemize}
\end{frame}

\begin{frame}
    \frametitle{Die Unendliche Geschichte - $\infty$}
               Downtime: 5 Tage mit mehreren Unterbrechung\\
               ETA: ---

               \begin{itemize}
                   \item Festplatte getauscht
                       \pause
                   \item L�uft.
               \end{itemize}
\end{frame}
\begin{frame}
    \frametitle{Die Unendliche Geschichte - $\infty$}
    Und das ganze jetzt in der Praxis ;)
\end{frame}
\end{document}
